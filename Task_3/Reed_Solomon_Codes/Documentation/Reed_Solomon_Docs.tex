\documentclass{scrartcl}
\usepackage{graphicx} % Required for inserting images

\title{BitSync Task - Reed Solomon Codes}
\author{Arjun Arunachalam}
\date{\textsc{April 2025}}

\begin{document}

\maketitle

\section{Introduction}
Reed-Solomon codes were developed by Iriving S Reed and Gustave Solomon when working at MIT Labs in 1960. They use an alphabet set of length $q = p^m,$ q being a prime power (i.e. the encoding set is a galois field) and encode the input at $n$ points; later interpolating it to a polynomial. Using this, errors are detected or corrected and \textbf{binary erasures} can be taken care of.

\section{Documentation}

\subsection{\texttt{SimulateChannel.py}}

\begin{itemize}
    \item Requirements
\begin{verbatim}
import numpy as np
import seaborn as sns
import matplotlib.pyplot as plt
\end{verbatim}

\item {Function to encode the text. Returns a list of encoded values. In this case, the encoding is just the number assigned to ASCII characters in Python.}
\begin{verbatim}
def encode_alphabet_naive(text):
\end{verbatim}

\item A custom encoding based on values modulo (3).
\begin{verbatim}
def encode_alphabet_custom(text):
\end{verbatim}

\item Function to decode. Here, decoding is based on the original function; i.e. Python assigned ASCII value to character.
\begin{verbatim}
def decode_alphabet(encoded_values):
\end{verbatim}

\item Returns a list of random values from the list $[1, n],$ with $n$ being the length of the pdist. The values are chosen based on corresponding probabilities in the pdist.
\begin{verbatim}
def random_out(pdist):
    return np.random.choice(np.arange(1, len(pdist) + 1), p=pdist)
\end{verbatim}
Example: If pdist = [0.1, 0.1, 0.8], output could be [1, 3, 3] as 3 has higher probability.

\item Plot the channel matrix as a heatmap.
\begin{verbatim}
def plot_channel_matrix(probab_matrx):
\end{verbatim}

\item Mean squared error function. Inputs actual, predicted are lists. Converted to numpy array to support index by index operations directly (actual - predicted = array of differences in values).
\begin{verbatim}
def mean_squared_error(actual, predicted):
    actual = np.array(actual)
    predicted = np.array(predicted)
    return np.mean((actual - predicted) ** 2)
\end{verbatim}

\item Normalize rows and columns of the probability matrix.
\begin{verbatim}
probab_matrx = probab_matrx / probab_matrx.sum(axis=1, keepdims=True)
probab_matrx = probab_matrx / probab_matrx.sum(axis=0, keepdims=True)
\end{verbatim}

\item Show results and error of custom encoding and ASCII encoding. To decide optimal encoding.
\begin{verbatim}
plot_channel_matrix(probab_matrx)
\end{verbatim}
\end{itemize}

\newpage

\subsection{\texttt{{Hamming\_Code\_Encoder\_Decoder.py}}}

\begin{itemize}
\item Requirements. Numpy
\item Define Generator matrix G. (4, 7) and parity check matrix H (3, 7).
\item Encoder. Comverts 4 bit to 7 bit. Accepts a 4 bit encoding as input. Conversion is done by taking dot product of encoding with a generator matrix.

\begin{verbatim}
def encode_batch(data_bits):
\end{verbatim}

\item Decoder. Takes in 7-bit input and converts it into the original 4-bit message. Procedure as follows.
    \begin{itemize}
        \item Get syndrome for error to bit mapping. Take dot product of parity check matrix and transpose of received code.
        \item Obtain error bit based on positions obtain in syndrom tuple.
        \item Copy the bit to a temporary variable, corrected.
        \item If the bit is errored, flip corrected.
        \item Append corrected to decoded array.
\end{itemize}  
\begin{verbatim}
def decode_batch(received_batch):
\end{verbatim}

\item Simulate a channel with gaussian noise. Initialise standard deviation of noise distribution. Array of length = codewords.shape (length of code) picked from the distribution using np.random.normal(). Add noise to the codeword array and return the clipped array (round off values to integral/binary bits) using np.clip and np.round.
\begin{verbatim}
def gaussian_channel(codewords, noise_std=0.5):
\end{verbatim}

\item Simulate a channel with uniform noise; i.e. bit flipping with probability $p.$ Induce flips with probability $p$ by choosing random integers in the range [1, codewords.shape], with probability flip\_prob. Return the error induced array by adding and taking mod 2 for binary. Can be extended to n-nary.
\begin{verbatim}
def uniform_channel(codewords, flip_prob=0.1):
\end{verbatim}

\item Return the bit error rate = number of errored bits/total number of bits.
\begin{verbatim}
def bit_error_rate(original, decoded):
\end{verbatim}

\end{itemize}

\newpage
\subsection{\texttt{Reed\_Solomon\_Encoder\_Decoder.py}}
\begin{itemize}
\item Requirements. \texttt{numpy, galois.}
\item Simulate binary erasure channel with erasure\_prob. An array of random indices is chosen with probability erasure\_prob using np.random.rand. Original codeword is copied to an array and chosen indices are replaced with '?' to indicate erasures. Returns the array.
\begin{verbatim}
def binary_erasure_channel(codeword, erasure_prob):
\end{verbatim}

\item Class for Reed Solomon Error Correction. Initialised variables and functions.
\begin{itemize}
    \item $m$ = Degree of field.
    \item GF = Galois Field of order $m$ for the prime $p = 2$ in this case.
    \item $n$ = Length of the codeword
    \item $k$ = Total length of the message.

    \item \texttt{def rs\_encode(self, message)} - First check for length of message. Fit the message into a polynomial in the GF by using \texttt{galois.Poly()} and store its range in a variable xs. The codeword will be the set of values at the points $\{ x_i \}$; stored in the variable codeword.

    \item \texttt{def rs\_decode(self, received)} - Decode the message by computing the value of the interpolated polynomial at the points $\{ x_i \}.$ Use Lagrange Interpolation inbuilt in galois library. xs = $\{x_i\}$ is known and input is the receieved code. Add zeroes in the message to account for erasures and to maintain length of decoded message.

    \item \texttt{def compute\_rate(self)} - Returns rate = $k/n.$ (Number of message bits/Total number of bits).
\end{itemize}

\item \texttt{main()} to call functions and display results.

\end{itemize}

\end{document}
